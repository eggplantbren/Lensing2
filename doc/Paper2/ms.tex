% mnras_template.tex
%
% LaTeX template for creating an MNRAS paper
%
% v3.0 released 14 May 2015
% (version numbers match those of mnras.cls)
%
% Copyright (C) Royal Astronomical Society 2015
% Authors:
% Keith T. Smith (Royal Astronomical Society)

% Change log
%
% v3.0 May 2015
%    Renamed to match the new package name
%    Version number matches mnras.cls
%    A few minor tweaks to wording
% v1.0 September 2013
%    Beta testing only - never publicly released
%    First version: a simple (ish) template for creating an MNRAS paper

%%%%%%%%%%%%%%%%%%%%%%%%%%%%%%%%%%%%%%%%%%%%%%%%%%
% Basic setup. Most papers should leave these options alone.
\documentclass[a4paper,fleqn,usenatbib]{mnras}

% MNRAS is set in Times font. If you don't have this installed (most LaTeX
% installations will be fine) or prefer the old Computer Modern fonts, comment
% out the following line
\usepackage{newtxtext,newtxmath}
% Depending on your LaTeX fonts installation, you might get better results with one of these:
%\usepackage{mathptmx}
%\usepackage{txfonts}

% Use vector fonts, so it zooms properly in on-screen viewing software
% Don't change these lines unless you know what you are doing
\usepackage[T1]{fontenc}
\usepackage{ae,aecompl}


%%%%% AUTHORS - PLACE YOUR OWN PACKAGES HERE %%%%%

% Only include extra packages if you really need them. Common packages are:
\usepackage{graphicx}	% Including figure files
\usepackage{amsmath}	% Advanced maths commands
\usepackage{amssymb}	% Extra maths symbols

\usepackage{microtype}

%%%%%%%%%%%%%%%%%%%%%%%%%%%%%%%%%%%%%%%%%%%%%%%%%%

%%%%% AUTHORS - PLACE YOUR OWN COMMANDS HERE %%%%%

% Please keep new commands to a minimum, and use \newcommand not \def to avoid
% overwriting existing commands. Example:
%\newcommand{\pcm}{\,cm$^{-2}$}	% per cm-squared

%%%%%%%%%%%%%%%%%%%%%%%%%%%%%%%%%%%%%%%%%%%%%%%%%%

%%%%%%%%%%%%%%%%%%% TITLE PAGE %%%%%%%%%%%%%%%%%%%

% Title of the paper, and the short title which is used in the headers.
% Keep the title short and informative.
\title[]
{Constraints on the masses of substructures in $x$ lens galaxies}
    
\author[Brewer and Lewis]{%
  Brendon~J.~Brewer$^{1}$\thanks{To whom correspondence should be addressed. Email: {\tt bj.brewer@auckland.ac.nz}},
  Geraint F. Lewis$^2$
  \medskip\\
  $^1$Department of Statistics, The University of Auckland, Private Bag 92019, Auckland 1142, New Zealand\\
  $^2$Sydney Institute for Astronomy, School of Physics, A28,
  The University of Sydney, NSW 2006, Australia}
% These dates will be filled out by the publisher
\date{}

% Enter the current year, for the copyright statements etc.
\pubyear{2016}

% Don't change these lines
\begin{document}
\label{firstpage}
\pagerange{\pageref{firstpage}--\pageref{lastpage}}
\maketitle

% Abstract of the paper
\begin{abstract}
We present inferences of the lens mass profile and source surface brightness
profile for a selection of galaxy-galaxy gravitational lens systems,
allowing for the possibility of substructures.
In the ``jackpot'' system (SDSSJ0946+1006) we find X,

\end{abstract}

% Select between one and six entries from the list of approved keywords.
% Don't make up new ones.
\begin{keywords}
gravitational lensing: strong --- methods: data analysis --- methods: statistical
\end{keywords}

%%%%%%%%%%%%%%%%%%%%%%%%%%%%%%%%%%%%%%%%%%%%%%%%%%

%%%%%%%%%%%%%%%%% BODY OF PAPER %%%%%%%%%%%%%%%%%%

\section{Introduction}

\citet{lensing2} presented a model of the prior information used to perform
inference from an image of a lensed galaxy. In this paper, we used a slightly
modified model.

\section{Prior Distribution}
The joint prior distribution for the hyperparameters, parameters, and the
data is summarised in Table~\ref{tab:priors}.

\begin{table*}
\begin{tabular}{|l|l|l|}
\hline
Quantity	&	Meaning		& Prior\\
\hline
{\bf Numbers of Blobs}\\
\hline
$N_{\rm src}$	&	Number of source blobs	& Uniform($0, 1, ..., 100$)\\
$N_{\rm lens}$	&	Number of lens blobs	& Uniform($0, 1, ..., 10$)\\
\hline
{\bf Source hyperparameters} ($\boldsymbol{\alpha}_{\rm src}$)\\
\hline
$(x_c^{\rm src}, y_c^{\rm src})$ & Typical position of source blobs & iid Cauchy(location=$(0,0)$, scale=0.1$\times${\tt imageWidth})\\
$R_{\rm src}$   &	Typical distance of blobs from $(x_c^{\rm src}, y_c^{\rm src})$ & LogUniform(0.01$\times${\tt imageWidth}, 10$\times${\tt imageWidth})\\
$\mu_{\rm src}$ &	Typical flux of source blobs	& $\ln(\mu_{\rm src}) \sim$ Cauchy(0, 1)$T(-21.205, 21.205)$\\
$W_{\rm max}^{\rm src}$ &		Maximum width of source blobs	& LogUniform(0.001$\times${\tt imageWidth}, {\tt imageWidth})\\
$W_{\rm min}^{\rm src}$ &		Minimum width of source blobs	& Uniform(0, $W_{\rm max}^{\rm src}$)\\
\hline
{\bf Lens hyperparameters} ($\boldsymbol{\alpha}_{\rm lens}$)\\
\hline
$(x_c^{\rm lens}, y_c^{\rm lens})$ & Typical position of lens blobs & iid Cauchy(location=$(0,0)$, scale=0.1$\times${\tt imageWidth})\\
$R_{\rm lens}$  &	Typical distance of blobs from $(x_c^{\rm lens}, y_c^{\rm lens})$ 	& LogUniform(0.01$\times${\tt imageWidth}, 10$\times${\tt imageWidth})\\
$\mu_{\rm lens}$&	Typical mass of lens blobs	& $\ln(\mu_{\rm lens}) \sim$ Cauchy(0, 1)$T(-21.205, 21.205)$\\
$W_{\rm max}^{\rm lens}$ &		Maximum width of lens blobs	& LogUniform(0.001$\times${\tt imageWidth}, {\tt imageWidth})\\
$W_{\rm min}^{\rm lens}$ &		Minimum width of lens blobs	& Uniform(0, $W_{\rm max}^{\rm lens}$)\\
\hline
{\bf Source Blob Parameters} ($\boldsymbol{\theta}_i^{\rm src}$)\\
\hline
$(x_i^{\rm src}, y_i^{\rm src})$ & Blob position & Circular(location=$(x_c^{\rm src}, y_c^{\rm src})$, scale=$R_{\rm src}$)\\
$A_i$  & Blob flux & Exponential(mean=$\mu_{\rm src}$)\\
$w_i$  & Blob width & Uniform($W_{\rm min}^{\rm src}$, $W_{\rm max}^{\rm src}$)\\
\hline
{\bf Lens Blob Parameters} ($\boldsymbol{\theta}_i^{\rm lens}$)\\
\hline
$(x_i^{\rm lens}, y_i^{\rm lens})$ & Blob position & Circular(location=$(x_c^{\rm lens}, y_c^{\rm lens})$, scale=$R_{\rm lens}$) \\
$M_i$  & Blob mass & Exponential(mean=$\mu_{\rm lens}$)\\
$v_i$  & Blob width & Uniform($W_{\rm min}^{\rm lens}$, $W_{\rm max}^{\rm lens}$)\\
\hline
{\bf Smooth Lens Parameters} ($\boldsymbol{\theta}_{\rm SIE}$)\\
\hline
$b$ & SIE Einstein Radius & LogUniform(0.001$\times${\tt imageWidth}, {\tt imageWidth})\\
$q$ & Axis ratio & Uniform(0, 0.95)\\
$(x_c^{\rm SIE}, y_c^{\rm SIE})$ & Central position & iid Cauchy(location=$(0,0)$, scale=0.1$\times${\tt imageWidth})\\
$\theta$ & Orientation angle & Uniform$(0, \pi)$\\
$\gamma$ & External shear & Cauchy$(0, 0.05)T(0, \infty)$\\
$\theta_\gamma$ & External shear angle & Uniform$(0, \pi)$\\
\hline
{\bf Noise Parameters} ($\boldsymbol{\sigma}$)\\
\hline
$\sigma_0$ & Constant component of noise variance & $\ln(\sigma_0) \sim$ Cauchy(0, 1)$T(-21.205, 21.205)$\\
$\sigma_1$ & Coefficient for variance increasing with flux &
$\ln(\sigma_1) \sim$ Cauchy(0, 1)$T(-21.205, 21.205)$\\
\hline
{\bf Data} ($\boldsymbol{D}$)\\
\hline
$D_{ij}$ & Pixel intensities & Normal$(m_{ij}, s_{ij}^2 + \sigma_0^2 + \sigma_1m_{ij})$
\end{tabular}
\caption{The prior distribution for all hyperparameters, parameters, and the
data, in our model. Uniform$(a, b)$ is a uniform
distribution between $a$ and $b$. LogUniform$(a, b)$ is a log-uniform
distribution (with density $f(x) \propto 1/x$, sometimes erroneously called
a Jeffreys prior) between $a$ and $b$. The notation $T(\alpha, \beta)$ after
a distribution denotes truncation to the interval $[\alpha, \beta]$. The
constant {\tt imageWidth} is the geometric mean of the image dimensions in
the $x$ and $y$ directions. {\color{blue} THIS WILL NEED TO BE UPDATED}
\label{tab:priors}}
\end{table*}


\section{Einstein Ring 0047-2808}

\section{The Jackpot}
\citep{sonnenfeldJackpot}
\citep{vegettiJackpot}

\section{Cosmic Horseshoe HST Image}


\section*{Acknowledgements}
It is a pleasure to thank Matt Auger (Cambridge) for valuable discussion and
providing the jackpot data. Barkana

This work was supported by a Marsden Fast Start grant from the Royal Society of
New Zealand. We also made use of resources provided by the Centre for
eResearch at the University of Auckland.

%%%%%%%%%%%%%%%%%%%%%%%%%%%%%%%%%%%%%%%%%%%%%%%%%%

%%%%%%%%%%%%%%%%%%%% REFERENCES %%%%%%%%%%%%%%%%%%

% The best way to enter references is to use BibTeX:

\bibliographystyle{mnras}
\bibliography{references} % if your bibtex file is called example.bib


%%%%%%%%%%%%%%%%%%%%%%%%%%%%%%%%%%%%%%%%%%%%%%%%%%

%%%%%%%%%%%%%%%%% APPENDICES %%%%%%%%%%%%%%%%%%%%%

%\appendix
%\section{Some extra material}

%%%%%%%%%%%%%%%%%%%%%%%%%%%%%%%%%%%%%%%%%%%%%%%%%%


% Don't change these lines
\bsp	% typesetting comment
\label{lastpage}
\end{document}

% End of mnras_template.tex
