% Every Latex document starts with a documentclass command
\documentclass[a4paper, 11pt]{article}

% Load some packages
\usepackage{graphicx} % This allows you to put figures in
\usepackage{natbib}   % This allows for relatively pain-free reference lists
\usepackage[left=3cm,top=3cm,right=3cm]{geometry} % The way I like the margins

% This helps with figure placement
\renewcommand{\topfraction}{0.85}
\renewcommand{\textfraction}{0.1}
\parindent=0cm

% Set values so you can have a title
\title{}
\title{Flexible trans-dimensional modelling of gravitational lenses}
\author{Brendon J. Brewer$^1$, David Huijser$^1$, and
Geraint F. Lewis$^2$}
\date{\small $^1$ Department of Statistics, The University of Auckland\\
Private Bag 92019, Auckland 1142, New Zealand\\
$^2$ Sydney Institute for Astrophysics\\
\vspace{1cm}
{\tt bj.brewer@auckland.ac.nz}}
% Document starts here
\begin{document}

% Actually put the title in
\maketitle

\abstract{Gravitational lensing is a powerful astrophysical tool for studying
the distribution of matter in galaxies.}

\section{Introduction}


\subsection{The hunt for dark substructures}
{\bf Review of what's been done in this area. How was it done, and what do
people believe?}

\section{Source and Lens Models}
Many different approaches have been applied to gravitational lens modelling.
Any such method will involve many choices, such as:
\begin{itemize}
\item Whether to use a simply parameterized (e.g. Sersic)
or flexible (e.g. pixellated)
model for the surface brightness profile of the source
\item Whether to use a simply parameterized or flexible
model for the mass profile of the lens
\item How to compute the results (e.g. optimization methods, Markov Chain
Monte Carlo)
\end{itemize}

All such choices involve tradeoffs between convenience and realism.


\section{The Cosmic Horseshoe}
We demonstrate the method on the Cosmic Horseshoe INT data.


\section{ER 0047-2808}
We demonstrate the method on the HST images of ER 0047-2808.


\end{document}

